% !TEX encoding = UTF-8 Unicode
%!TEX root = thesis.tex
% !TEX spellcheck = en-US
%%=========================================
\chapter{Discussion}

\section{Dataset matching}


\begin{itemize}
	\item How good is the matching in the three different cases. Rotation and scale. Try to quantify error.
	\item How to improve matching
	\item How much does rotation and rebinning of data change it
	\item How good is the matching of the datasets
	\item How to improve it
	\item How to do the final step
\end{itemize}

\subsection{Step 1}

It is difficult to quantitatively determine how good the matching of the EDX dataset in the SPED dataset is, and there are many factors that contribute to the inaccuracy of the matching. Because the rotation angle was only changed by 0.4\% when being iterated over, it appears that the rotation angle is not a big source of inaccuracy. A bigger potential problem is the apparent discrepancy in scaling the EDX and SPED datasets. This could be due to the scaling factors being wrong, but can also be purely because of the preprocessing steps before image matching was performed.

As seen in \cref{fig:D,fig:E} (and even clearer in \cref{fig:zeta_area1,fig:zeta_area2} for the untreated sample), there is an overlapping region of approximately \SI{2}{\nano\meter} between layers of different composition. In this overlapping region, the EDX dataset will show Ga and As peaks, although not as high as in the nanowire, and this overlap region will therefore appear bright in the EDX image. In the SPED dataset there is also an overlapping region in which the diffraction spots from the nanowire gradually lose intensity, which was measured (using HyperSpyUI) to be approximately \SI{10}{\nano \meter}. This difference of lengths of the overlapping regions could cause the preprocessing steps to include slightly different regions of each datasets. As only these selected regions will appear bright in the images used for template matching, the sizes of the nanowires would be different, even though the scaling factor is correct.

However, in \cref{fig:edxspedorig,fig:edxspedrot}, to the right of the nanowire there is a clear difference between the EDX and SPED regions. This might also be due to the differences in overlap regions, but it gives a strong indication that the scaling factor might not be correct.

There are several possible improvements that could be made to the algorithm in order to improve these results. Firstly, a stronger filter could be placed on the images in order to not include any part of the overlap region from either dataset. Secondly, different elements in the sample, not just the GaAs in the nanowire, could be used to make the images more similar. For instance, if there are regions that have consistent diffraction patterns and compositions, these could be given a different color in the images. In this way, up to three different parts of the sample (one for each independent color) could be used to match the images together.

Improvements could possibly also have been made to the experimental setup. The current matching was possible due to the nanowire being a recognizable feature in the sample; however, the diffuseness of the edges appears to decrease the accuracy. It is difficult to change this behavior of the edges, but introducing a new feature with sharper edges might reduce the number of error sources. In addition, the accuracy of the scaling factors ought to be investigated and, if necessary, corrected to give the right values.

In the final product, the SPED dataset has been rotated and the EDX dataset re-binned. These procedures might also introduce new errors that the person analyzing the datasets must be aware of. The rotation algorithm, though it is still in a preliminary stage and might contain errors, does not change any of the information in the datasets and should therefore be regarded as safe. The re-binning algorithm, on the other hand, can change the EDX spectrum significantly, and especially if the dataset is re-binned by a factor much larger or much smaller than unity. This is especially important to keep in mind for the smaller datasets, that have a much higher resolution than the SPED dataset, and would therefore need to be down-sampled by a large factor,  resulting in a loss of information. If the purpose of the analysis is merely to qualitatively look at the composition and structure of a region in the sample, this information loss might be acceptable. However, for accurate quantitative results, one must be very careful to blindly trust the data.

\subsubsection{Step 2}

The matching of all the datasets in the HAADF overview image gave a good result for most of the datasets. The results have been verified by eye, and so there might be errors and inaccuracies that are not easily seen. For the purpose of knowing where in the sample the different datasets have been taken from, visual confirmation of the accuracy is good enough. However, if these locations are to be used to cut out portions of the SPED dataset in order to compare it to the EDX datasets, small discrepancies might lead to wrong conclusions about compositions and structures.

The datasets that were not correctly located in the overview image, or were not part of it, gave matching values that were 20-60\% lower than the datasets assumed to be the best match. This distinction is very clear in the untreated image (see \cref{fig:nonheated-matching-values}), but dataset A in the heated image (see \cref{fig:heated-matching-values}) gave a rather high relative matching value, even though the dataset is not part of the image. This could easily lead to false conclusions about the location of the dataset, especially if it is not immediately clear from visual inspection that the algorithm's answer is wrong.

This problem is believed to be improved by changing the way the ideal contrast level is determined. Instead of finding the overall best contrast level for all the datasets, this level can be determined for each dataset individually. This improvement is believed to simultaneously increase the accuracy of the actual matching and increase the accuracy in which datasets that are not included in the overview image are identified as such.

Even though improving the algorithm is believed to give considerably better results, there are also changes that could and should be made to the data acquisition at the microscope. The algorithm was able to give accurate results for most of the datasets because the survey images were large enough to be uniquely identified in the overview image. However, the accuracy of the location of dataset F in the untreated sample was impossible to verify due to the survey image being too small and not containing enough distinct features. The TEM operator should therefore remember to always make sure the survey images are taken over large enough areas, or over areas which includes clear distinguishable features.

\subsubsection{Step 3}

The final step in the procedure would be to use the locations of these EDX datasets to find the equivalent positions in the SPED dataset. This was done successfully for only one of the datasets (dataset F) - the rest resulted in mismatching array dimensions between the rotated SPED dataset and the re-binned EDX dataset. This is believed to be due to the small size (low number of pixels) of the other datasets, as datasets C and F are much larger than the rest. As already discussed, the high resolution of these remaining datasets compared to the SPED dataset might lead one to conclude that this final step is unnecessary, especially if accurate quantitative results are required.

\section{Determination of $\zeta$-values}

When calculating the $\zeta$-factors, there are two main uncertainties: The thickness of the sample and the background subtraction. In his thesis, Garmannslund estimates a relative error in the thickness of 5.26\%. The errors due to the background subtraction are more difficult to quantify. Errors due to the background windows are believed to be lower for the peaks at higher energies due to the flatter background, as seen in \cref{fig:spectrum-with-info}. %The integration window width is believed to not have accounted

Other uncertainties: Probe current, integration window, ...?

The spread in $\zeta$-values between the different regions in the samples are not significant, and are well within the uncertainty of the thickness, except for Ga and As. The large deviance between the $\zeta$-values for these elements between the untreated and heat-treated samples can have been caused by several different factors. Shadowing, probe currents, sample thickness%Differences in shadowing due to the different tilt levels in each setup will have contributed, but is believed to not 

The $\zeta$-factors were compared to the $k$ in order to partially verify the values. As the $k$-factors are known to be less accurate than the calculated $\zeta$-factors, the results in \cref{fig:zeta-k-comparison} are good for the lighter elements. However, the results for Pd and especially Au show very high differences between the two methods. It is again difficult to explain what the cause of this is. ???

How to make them better:
Verifying by thickness??

\section{Quantification}

The quantification of the untreated sample gave good results in both regions. The inclusion of absorption correction gave only very minor differences in compositions, but that is to be expected when the compositions are this well-defined. The fact that the $\zeta$-method gave results that matched the known compositions better than the CL-method did is an indication of the validity of this method, but not a validation. Furthermore, both methods produced clearly wrong peaks, ...
%When the results are visualized as in \ref{fig:zeta_area1,fig:zeta_area2} they are clearly identifiable as outliers, but if only the mean value is considered

%When analyzing the results from the heat-treated sample, it is important to keep in mind the results from the untreated sample and not just blindly trust the presented data. In particular, the overlap regions visible in \cref{fig:zeta_area1,fig:zeta_area2} must be 

The quantification of the heat-treated sample has not given as clear results as one might have hoped, and it is important to keep in mind the quantification process and not just blindly trust the presented data. For instance, the transition regions between the three layers in areas D and E are calculated to contain up to 10\% Au even though none of the three layers shows more than 2\% Au. The small presence of Au in the results is likely to occur due to spurious X-rays that are scattered to high angles and then produce characteristic X-rays from Au-atoms at other locations in the sample. This hypothesis is strengthened by looking at the EDX spectrum from the nanowire in the heat-treated sample, in which there is a small $Au_M\alpha$-peak (as well as a small $Pd_L\alpha$-peak). However, there is no peak of significant height to be seen for the $Au_L\alpha$-line, which lies at a much higher energy than the $Au_M\alpha$-line. +++

Why the quantification shows more Au in the transition regions than in the main layers is unknown (++discuss this), but the conclusion is nevertheless that there is likely very little or no Au in either of the layers. The accuracy of this conclusion is likely to have been verified by using the $Au_{L\alpha}$-line instead, as the higher energy makes it less likely to give counts because of spurious X-rays, but the high overlap with $Ge_{K\alpha}$ makes this difficult to do in practice.

If one is to conclude that there is no Au in areas D and E, the reported amount of Pd must also be adjusted. As mentioned, the nanowire showed a small $Pd_{L\alpha}$-peak as well, at an energy close to that of $Au_{M\alpha}$, likely also a product of spurious X-rays. This will have increased the reported amount of Pd also in the Pd-rich regions, and skewed the ratios in favor of Pd. In order to estimate how much higher the reported amounts of Pd are compared to the actual amounts, one can look at the amount of Pd detected in the nanowire, that is, the inner regions of areas D and E. If the nanowire is assumed to consist of only Ga and As (or verified to do so from the diffraction patterns), there should be no Pd detected. As presented in \cref{tab:D-composition,tab:E-composition}, the $\zeta$-method gave 0-7\% and 0-10\% Pd in the nanowire in areas D and E, respectively. Subtracting an amount of up to these values from the Pd-composition and attributing it to the other elements might therefore give a more accurate idea of the actual composition of the material. However, the nanowire also shows a composition of 0-5 and 0-7 \% Ge in areas D and E, which was assumed to not be due to spurious X-rays because of the higher energy of the $\mathrm{Ge_{K\alpha}}$-peak. If this result is due to spurious X-rays as well, the subtraction of Pd must be reduced accordingly, and the composition will therefore not change much. 

\cref{tab:D-composition,tab:E-composition} gives three different results for the Pd-composition in these areas, and it is not obvious which is to be trusted to be the most correct. As discussed above, the compositions denoted $\mathrm{Pd}_1$ are likely to be too high due to spurious X-rays. The Pd$_2$-compositions calculated without $\zeta$-absorption give results corresponding fairly well to a subtraction of 7-10\% pp Pd. However, it must be noted that if one assumes no Au in either regions, the up to 10\% reported Au must be allocated to the other elements, making the $\mathrm{Pd}_2$-composition slightly higher than the results in the tables, both with and without absorption correction. These results give no answer to whether the absorption correction gives more accurate results for these areas, nor to which peaks ought to be used when 

The areas 1,2 and 3 in \cref{fig:BCF} showed similar results as for areas D and E, and should therefore be analyzed in the same way. In \cref{fig:BCF1}, one may therefore assume that the Ge:Pd-ratio in areas 1 and 2 ought to be skewed from about 43:57 (for the $zeta$-method) to about 50:50, and likewise in area 3, except that also the Ga- and As-compositions should increase. This hypothesis is strengthened by looking at \cref{fig:BCF2} for areas and 1 and 2, which contain no Ga or As, as \cref{fig:BCF1} shows. For these regions, the results therefore ought to be accurate also for Ge, despite the overlap between peaks. It is now interesting to note the big differences caused by the inclusion of absorption correction to the $\zeta$-method. Without absorption correction, the results are very similar as the ones in \cref{fig:BCF1}, but absorption correction gives results that match very well the hypothesis of spurious X-rays giving a higher Pd-composition. If this hypothesis can be proven to be correct through analysis of the diffraction patterns, it gives a strong reason for why absorption correction can be necessary to get accurate results.

\section{Comparison to theory}

From the literature, \cite{nanowire-theory} (++++ this part should reference introduction instead) there are some candidate compositions that one would expect to see also for this sample. Figure \cref{fig:...} shows the expected compositions in different layers (++++ experiment conditions). From this, one might expect to find PdGe in areas 1 and 2 in \cref{fig:D,fig:E}, GaAs with diffused Ge in the inner regions of areas D and E (\cref{fig:DE}), and perhaps the expected intermediate phase Pd$_4$GaAs in other areas.

In areas 1 and 2, the $\zeta$-method with absorption correction when using the Ge$_{L\alpha}$-peak corresponds very well to the theory, and as discussed, a similar result as when using the Ge$_{K\alpha}$-peak if the reported Ge:Pd-ratio is skewed as a result of spurious X-rays. This hypothesis is further strengthened if the inner region of the nanowire does consist of GaAs with diffused Ge, as this result would indicate that the reported presence of Ge is not just a result of spurious X-rays, in turn validating the skewing of the Ge:Pd-ratio.

For the lower and upper regions in areas D and E, the results are not as easily interpreted because the results from using the Ge$_{L\alpha}$-line can not be trusted. The lower regions appear to consist of Ge and Pd in a ratio Pd:Ge between 3:2 and 5:3, along with traces of Ga and As in a ratio between and 1:20 and 1:5 to Pd. However, if analysis of the diffraction patterns from areas 1 and 2 reports the phase to be GePd, the same logic as above ought to be used here as well, and the Pd:Ge ratio should be skewed to between 1:1 and 5:4, depending on how much of the reported Pd and Au is attributed to each of the other elements. This would also mean that the ratios of Ga and As to Pd should increase slightly.
%However, if the results from area 1 and 2 indicate that absorption correction gives more accurate results, the Ge:Pd-ratio might be skewed to 3:2 or even 2:1, by using the parenthesized values for Pd$_2$ in \cref{tab:D-composition,tab:E-composition}, and therefore increasing the compositions of the other elements.
In the upper regions the composition is Pd:Ga:As:Ge in a ratio Pd:GaAs:Ge:Au approximately between 22:4:2:1 and 12:4:2:1. Again, skewing the ratio and attributing the Au-composition to the other elements will make the expected intermediate phase Pd$_4$GaAs a possibility, except for the presence of Ge. 